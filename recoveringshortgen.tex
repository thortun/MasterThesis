\section{Recovering the Generator}
Given a generator of a principal ideal, the task is to make it as short as possible. As discussed \improvement{check that it actually is discussed}, if we can recover a short enough generator for the ideal, we are able to transform this generator to a short vector in the corresponding lattice. Creamer et.al. showed in \cite{Recover Short Gen} that there exists a polynomial algorithm to do this. We will present the main machinery for this algorithm and then present the details.

\begin{theorem}
\label{Thm: Recover Short Gen}
    Let \(D\) be a distribution over \(\QQ(\zeta)\) with the property that for any tuple of vectors \(\vec{v}_1,\dots ,\vec{v}_{\varphi(m)/2 - 1}\in\RR^{\varphi(m)/2}\) of Euclidian norm 1 that are orthogonal to the all-1 vector \(\vec{1}\), the probability that \(|\langle\emph{Log}(g), \vec{v}_i\rangle| < c\sqrt{m}\cdot (\log m)^{-3/2}\) holds for all \(i\) is at least \(\alpha\), where \(g\) is chosen from \(D\) and \(c\) is a universal constant. There there is an efficient algorithm that given \(g' = g\cdot u\), where \(g\) is chosen from \(D\) and \(u\in C\) is a cyclotomic unit, outputs an element of the form \(\pm\zeta^j g\) with probability at least \(\alpha\).
\end{theorem}
In addition to this theorem, we state a proposition that helps to solve a lattice problem. A very simple algorithm to solve BDDP is, on input basis \(\mathcal{B}\) and a target vector \(\vec{t}\), is called the round off algorithm and it simply outputs \(\mathcal{B}\cdot\lfloor({\mathcal{B}^\vee})^T\cdot\vec{t}\rceil\).
\begin{proposition}
\label{Prop: Round off Algorithm}
    Let \(\Lambda\) be a lattice over \(\RR\) with basis \(\mathcal{B}\), and let \(\vec{t} = \vec{v} + \vec{e} \in\RR^n\) for some \(\vec{v}\in\Lambda, \vec{e}\in\RR^n\). If \(\langle\vec{b}_j^\vee, \vec{e}\rangle\in\left[-\frac{1}{2},\frac{1}{2}\right)\) for all \(j\), then on input \(\vec{t}\) and basis \(\mathcal{B}\), the round-off algorithm outputs \(\vec{v}\).
\end{proposition}
\begin{proof}
Because \(\vec{v}\) is in the lattice have have that \(\vec{v} = \vec{z}\cdot\mathcal{B}\) for some integer vector \(\vec{z}\). Now, multiplying from the right on \(\vec{t} = \vec{v} + \vec{e}\) by \({(\mathcal{B}^\vee)^T}\) yields 
\begin{align*}
    \vec{t} {(\mathcal{B}^\vee)^T} = \vec{z} + \vec{e}\cdot{(\mathcal{B}^\vee)^T}
\end{align*} On the assumption on \(\langle\vec{b}_j^\vee, \vec{e}\rangle\) we get that
\begin{align*}
    \lfloor\vec{t} {(\mathcal{B}^\vee)^T}\rceil = \lfloor \vec{z} + \vec{e}\cdot{(\mathcal{B}^\vee)^T} \rceil = \vec{z}
\end{align*}
because \(\vec{z}\) is an integer vector.
\end{proof}
Now, looking hard at Theorem \ref{Thm: Recover Short Gen}, there are two main aspects of this result. There is a lattice, called the \emph{log-unit lattice} which will be defined later, that is related to orthogonality of \(\vec{1}\). We see elements from this lattice in the inner product in Theorem \ref{Thm: Recover Short Gen}. In the same inner product, we recognize a bound for an inner product, which relates to the round-off algorithm, Proposition \ref{Prop: Round off Algorithm}. If we can achieve this bound with non-negligible probability (related to \(\alpha\)), then we can solve BBDP. In this chapter we discuss the various aspects of this method, as described in \cite{Recover Short Gen}.
\subsection{Some More Background}
We now regard a number field \(K = \mathbb{Q}[\zeta_m]\), which can be embedded into \(\mathbb{C}\). The embeddings \(\sigma_j:\omega\mapsto\omega^j\) come in conjugate pairs, I.e. \(\sigma_j = \compconj{\sigma{-j}}\). Since we are mainly concerned by magnitudes, we index the embeddings over the multiplicative quotient group \(G = \ZZ_m^\vee/\{\pm 1\}\). We can then define the logarithmic embedding
\begin{definition}
The \emph{logarithmic embedding} is defined as
\begin{equation*}
    \begin{split}
        \emph{Log}:&\; \QQ[\zeta_m]\rightarrow \RR^{\varphi(m)}\\
                   & a\mapsto (\log |\sigma_j(a)|)_{j\in G}
    \end{split}
\end{equation*}
so any element \(a\) is mapped to a vector in \(\RR^{\varphi(m)}\). \improvement{did we talk about geometry?} It can be shown, by the Dirichlet Unit Theorem \improvement{talk about this?} that the image of Log under \(\OO_K^\vee\) is a full rank lattice of the linear subspace of \(\RR^{\varpi(m)}\) orthogonal to \(\vec{1}\). We refer to this lattice as the \emph{log-unit lattice}.
\end{definition}
Moreover, we want some cyclotomic units my man
\begin{definition}[Cyclotomic Units]
Let \(A\) be the multiplicative group generated by 
\begin{align*}
    \left\{\pm\zeta_m, 1 - \zeta_m^s\suchthat 1<s\leq m - 1  \right\},
\end{align*}
then the group of \emph{cyclotomic units} is defined as
\begin{align*}
    C = A\cap\OO_{\QQ[\zeta_m]}.
\end{align*}
\end{definition}
The group of cyclotomic units is the units of \(\QQ[\zeta_m]\). These units are hard to determine for general number fields\cite{Intro To Cyclotomic Fields}. We also have a nice expression for the generators of \(C\).
\begin{proposition}
Let \(m\) be a prime power, and define \(b_j := z_j/z_1 = (\zeta^j - 1)/(\zeta - 1)\). The group of cyclotomic units \(C\) is generated by \(b_j,\;\;j\in G\backslash \{1\}\)
\end{proposition}
\begin{proof}
See \cite{Intro To Cyclotomic Fields} probably. \needtodo{fix}
\end{proof}
We can notice that Log \(C\) is a sublattice of Log \(\OO^\vee\). In the algorithm, we find elements in Log \(C\). This might seem like a problem, but it is conjectured that \(\OO^\vee = C\) in many cases. It is shown for many power of 2 \(m\)-s \improvement{reference}.\par
To use the theorem, we need a bound of the norm of the basis elements of Log \(C\). Since \(b_j\) generate \(C\), and noticing that Log \(b_j = \) Log \(b_{-j}\) \improvement{Argue for this}, we define
\begin{align*}
    \vec{b}_j = \text{Log }b_j\quad j\in G\backslash\{ 1\}
\end{align*}
These \(\vec{b}_j\) form a basis for Log \(C\). Now, if we can bound the norm of the dual-vectors \(\vec{b}_j^\vee\), we can use the round-off algorithm to solve the BDDP in this lattice. This is essentially \cite{Recover Short Gen}. The following result does just that.

\begin{proposition}
    Let \(m= p^k\) be a prime power \(p^k\) and let \(\{\vec{b}_j^\vee\}_{j\in G\backslash \{1\}}\) be the dual basis to \(\{\vec{b}_j\}_{j\in G\backslash \{1\}}\). Then all the norms of \(\vec{b}_j^\vee\) are equal and
    \begin{align*}
        ||\vec{b}_j^\vee||^2 \leq 2k|G|\inverse\cdot(l(m)^2+O(1)) = O(m\inverse\cdot\log ^3 m).
    \end{align*}
\end{proposition}
We need a bit more machinery to prove this proposition. Recall that 
\begin{align*}
    b_j = z_j/z_1.
\end{align*}
We therefore define
\begin{align*}
    \vec{z}_j := \text{Log }z_j = \vec{b}_j + \vec{z}_1
\end{align*}
so that, for instance, \(\vec{z}_2 = \text{Log }(\zeta^2 - 1)\). Next, set up a \(G\times G\)-matrix \(\Vec{Z}\) by letting the \(j\)-th column of \(\vec{Z}\) be \(\vec{z}_{j\inverse}\). It looks something like this:
\begin{align}
    \vec{Z} = \left(\begin{tabular}{cccc}
           \(\log |\zeta^{1\cdot 1\inverse}-1|\) & \(\log |\zeta^{1\cdot 2^{-1}}-1|\) & \(\dots\) & \(\log |\zeta^{1\cdot k\inverse}-1 |\)\\
           \(\log |\zeta^{2\cdot 1\inverse}-1|\) & \(\log |\zeta^{2\cdot 2^{-1}}-1|\) & \(\dots\) & \(\log |\zeta^{2\cdot k\inverse} -1|\)\\
           \(\vdots\) & \(\vdots\) & \(\ddots\) & \(\vdots\) \\
           \(\log |\zeta^{k\cdot 1\inverse}-1|\) & \(\log |\zeta^{k\cdot 2\inverse}-1|\) & \(\dots\) & \(\log |\zeta^{k\cdot k\inverse}-1|\)
        \end{tabular}\right).
\end{align}
We can notice that this is exactly the \(G\)-circulant matrix associated with \(\vec{z}_1\) \improvement{really?}. For each eigenvector of \(\chi\in\hat{G}\), let \(\lambda_\chi = \langle \vec{z}_1, \chi\rangle\) be the corresponding eigenvalue. We then have that
\begin{proposition}
    For all \(j\in G\backslash\{ 1 \}\) we have that 
    \begin{align*}
        ||\vec{b}_j^\vee|| = |G|\inverse\cdot\sum\limits_{\chi\in \hat{G}\backslash \{ 1 \}}|\lambda_\chi |^2
    \end{align*}
\end{proposition}
\begin{proof}
Hei
\end{proof}
Now we have a bound for the length if the lattice vectors of the log-unit lattice. We have now the theory needed to prove Theorem \ref{Thm: Recover Short Gen}.
\begin{proof}
We are given an element \(g' = g\cdot u\). Now, applying the logarithm on both sides yields
\begin{align*}
    \text{Log }g' = \text{Log }g + \text{Log }u.
\end{align*}
By the assumptions of the distribution \(D\), the output of the round-off algorithm is correct, I.e. \(\text{Log }u\in\text{Log }C\). Next, we find, by linear algebra, coefficients \(a_j\) such that 
\begin{align*}
    \text{Log }u = \sum a_j\vec{b}_j.
\end{align*}
Since \(\vec{b}_j\) form a basis this can be done efficiently. Next, compute
\begin{align*}
    u' = \prod b_j^{a_j}.
\end{align*}
Since the kernel of the logarithmic embedding is the roots of unity \(\zeta\), we get that \(\text{Log }u' = \text{Log }u\) and conclude that \(u' = \pm\zeta^j u\) for some sign and some \(j\). Therefore we have that \(g'/u' = \pm\zeta^j u\) for some, possibly different, sign and \(j\). This finishes the proof.
\end{proof}
If we can find some bounds on certain lattice-vectors, then we are pretty golden. We therefore define the \emph{covering radius} of the lattice.
\begin{definition}[Covering Radius]
    The \emph{covering radius}, with respect to the \(p\)-norm, of a lattice \(\Lambda\) is defined as
    \begin{align*}
        \mu^{(p)}(\Lambda) = \max_{\vec{v}\in\emph{span}\Lambda}\;\min_{\vec{x} - \vec{v}}||\vec{x} - \vec{v}||_p.
    \end{align*}
\end{definition}
Geometrically, this is the smallest radius required to cover the whole lattice with balls centered around each lattice point. Now we introduce some notation and variables to get some bounds. 
\begin{proposition}
    Let \(H\) be the subspace of \(\RR^{\varphi(m)/2}\) spanned by Log \(\OO^\vee\). For any \(g\in R\), let \(\mathcal{I} = g\OO\) be the ideal generated by \(g\). Denote \(\vec{g} = \text{Log }g\) and write it as \(\vec{g} = s\vec{1} + \vec{g}_H\) where \(\vec{g}_H\in H\)\improvement{show that his is possible}. Then there exists an efficient algorithm that, given any \(h_H\in g_H + \text{Log }C\), outputs a generator \(h\) of \(\mathcal{I}\) such that 
    \begin{align*}
        ||h|| \leq \sqrt{\varphi (m)}\exp(||\vec{h}_H||_\infty ) \cdot S(\mathcal{I})
    \end{align*}
    where \(S(\mathcal{I}) = N(\mathcal{I})^{1/\varphi(m)}\) is the dimension-normalized norm.
\end{proposition}
Note that we have the subspace \(H\subseteq \OO\), which is the space orthogonal to \(\vec{1}\). Therefore we can decompose any vector \(g\in\OO\) as an element in \(H\) plus an element in the space spanned by \(\vec{1}\). Hence, the decomposition \(\vec{g} = s\vec{1} + \vec{g}_H\) always exists. Now we proceed with the proof.\needtodo{Make sure this is correct argument.}
\begin{proof}
    We begin by computing \(\vec{u} := h_H - g_H\) which is an element in \(\text{Log} C\) by definition. Then compute \(a_j\in\ZZ\) such that \(\vec{u} = \sum a_j\vec{b}_j\) and output \(h = \prod b_j^{a_j}\). Now, since \(\vec{h} := \text{Log }h = \text{Log }g + \vec{u} = s\vec{1} + h_H\) we have that
    \begin{align*}
        ||h||^2 = \sum\limits_{i\in\ZZ^\vee_m}|\sigma_i (h)|^2 \leq \varphi(m)\cdot\exp(||\vec{h}||_\infty)^2 = \varphi(m)\exp (||h_H||_\infty)^2\cdot S(\mathcal{I}
        )^2
    \end{align*}
    where the first inequality follows by
    \begin{align*}
        \text{Arguemnt here}
    \end{align*}
    and the last equality follows because
    \begin{align*}
        N(\mathcal{I}) = N(g) = \prod\limits_{i\in\ZZ^\vee_m}\sigma_i (g) = \prod\limits_{i\in G}|\sigma_i(g)|^2 = \exp (2\langle\vec{g}, \vec{1}\rangle) = \exp (s\cdot\varphi (m)).
    \end{align*}
\end{proof}
\subsection{Discussion}
The main machinery of Theorem \ref{Thm: Recover Short Gen} relies on some constraints on a distribution. This is discussed in \cite{Recover Short Gen} where it is proven that many natural distributions satisfies the conditions of the theorem. However, a crypto scheme which uses a distribution which doesn not is not as vulnerable to this attack. However, one might not want such a distribution as this can possibly make the system slow or weak in other respects \needtodo{find claims to back this up}.\par
In addition, the algorithm described in Theorem \ref{Thm: Recover Short Gen} only works for inputs \(g' = g\cdot u\) for a cyclotomic unit \(u\), whereas we want the algorithm to work for general units in \(\OO^\vee\)\improvement{correct right?}. As discussed in \cite{Recover Short Gen} this should not be a problem. It is conjectures that the index of units over cyclotomic units is small, and there are well backed heuristics and computation that show this \needtodo{references}. If the index is small, we can enumerate over the cosets, which there are few of, and run the algorithm a number of times equal the index. This is the only heuristic in this algorithm.\par