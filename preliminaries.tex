\section{Preliminaries}
\label{Section:Preliminaries}
Let \(a = (a_1,\dots ,a_n) \in\mathbb{R}^n\) be a vector, then denote by
\begin{align*}
    ||a||_l = \left(\sum\limits_{i = 1}^l |a_i|^n\right)^\frac{1}{l}
\end{align*}
the \(l\)-norm. We will almost always use the 2-norm because it is the best norm.\par
We use the notation \(\lceil a\rfloor\) to denote the closest integer to \(a\).\par
For problems \(A\) and \(B\), the notation \(A\leq B\) will be used to indicate that \(A\) is at least as hard as \(B\). In other words \(A\leq B\implies A \text{ reduces to }B\). Two equivalent problems are denoted \(A=B\). \par
We talk a lot about rings of integers. Here it is helpful to think about certain elements as denominators. Throughout this thesis we will denote these elements by \(d\) and keep the notation consistent throughout.